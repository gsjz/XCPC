\subsection{后缀自动机}

\lstinputlisting{String/automaton/suffix_automaton.cpp}

\subsubsection{备注}

一般来说,和 dag 相关的,是考虑在上面构建字符串。
这个时候 dag 起到一个压缩结点数量的作用。
当两种路径跑完,构建出一个串了,
结果后续能再添加的后缀完全一致,则没必要分别维护。

这种问题里,如果懒得再对 dag 或者 parent 树重新拓扑排序,
那么可以调用代码里 ord 数组的顺序,
即针对每个结点 len 桶排序后的结果来从小往大做,反之同理。

要注意,千万不能以为默认编号蕴含拓扑序。
因为每次增量构造时,是可能增加克隆结点的,
这些结点可能会成为一些已经建好的点的父亲。

\subsubsection{后缀树}

将字符串翻转后,跑 SAM,则 parent 树就是原串的后缀树。

可以这样考虑:parent 树的非克隆结点,对应一个完整的前缀。(对于翻转前字符串,则为完整后缀。)

对于两个非克隆结点的 lca 处的 len,其实就是它们的最长公共后缀。(对于翻转前字符串,则为两个完整后缀的最长公共前缀。)


