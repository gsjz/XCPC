\subsection{预备节:WQS 二分}

\subsubsection{传统方法}

\href{https://www.luogu.com.cn/problem/P2619}{P2619 [国家集训队]Tree I - 洛谷}

洛谷题解质量有点差,很多都看不出和 wqs 二分有什么关系。

这样思考,设 $g_x$ 是恰选中 $x$ 条白边的最小权。那么 $f(x) := g_x - \lambda x$ 就是给这 $x$ 条白边的权值减去 $\lambda$ 的惩罚代价后的权值,那么 $g_x-\lambda x$ 的最小值在什么地方取到?这等价于提前给所有白边都减去 $\lambda$,然后跑 mst。这有什么意义呢?

考虑 $(need-1, g_{need-1})$ 和 $(need, g_{need})$,当 $\lambda$ 恰好等于这两个点的斜率时,如果能有下凸性,并且同取最小值的时候可以取 $x$ 大的,这时候就会取到后面的那个作为最小值的 $x$ 点(如果取到 $>need$ 的其实同理可以往回取)。

当 $\lambda$ 大于这两个点的斜率时,$< need$ 的部分 比 大于等于 $\ge need$ 的部分不优。

当 $\lambda$ 大于这两个点的斜率时,$< need$ 的部分 比 大于等于 $\ge need$ 的部分更优。

所以二分这个惩罚代价,使得 $p = \max\{pos : f(pos) = \min_{x}\{f(x)\}\}$ 位于 $\ge need$ 部分,那么必然通过调整使得 $f(need) = f(p)$。


\lstinputlisting{Convex_Optimization/monge/wqs_binary_search.cpp}

\subsubsection{通用方法}

下面以二维 wqs 二分的题目为例,介绍一种不需要知道最小值横坐标,也就不需要处理共线问题的写法。

\href{https://codeforces.com/problemset/problem/1799/F}{CF1799F}

对于两维分别凸的函数,它的性质并不像一维的那么好。比如说对于一维情况,点 $x$ 的斜率自然地表述成 $x$ 和 $x-1$ 对应的点所确定的直线的斜率;对于二维呢?是考虑 $(x,y)$ 与 $(x-1,y)$ 与 $(x,y-1)$ 构成的切平面吗?然而这个切平面未必能确定其与 $(x-1,y-1)$ 的关系。另外,在一维的时候我们遇到的共线问题,在高维会变得更复杂。总之这个升维并不是显然的。

这里需要引入一个不太容易想得到的对偶函数(Legendre 变换):对于上凸的 $f(x)$,令

$$
g(k) := \max_{x}\left\{ f(x)-k(x-a) \right\}
$$

则可证明 $g(k)$ 是下凸函数,且最小值在 $k = f'(a)$ 时(离散情况则取离散定义,可能是一个区间,比如说 $\left[ f(x+1)-f(x),f(x)-f(x-1) \right]$ 中每一个 $k$)取到,即 $f(a)$。

这个操作的直观是,对于一个 fixed 的 $k$,在上凸壳的不同的点去做这个斜率的直线,交到 $x=a$ 的位置中,取最大的。实际上我们会发现,这个最大位置实际上取的就是直线与上凸壳的切点。

只有整个上凸壳的切线恰好在经过 $(a,f(a))$ 时,这个最大值才会取到下界 $f(a)$。为什么关于 $k$ 会下凸呢?因为随着切点远离 $x$,$\text{d}k$ 乘上这个距离会越来越大。

上面这种想法可以扩展。对于两维联合(这里只是另一维为常数意义下的话应该不太对的)上凸的 $f(x,y)$,现在想要单点求值,$f(a,b)$。那么只需要考虑由 $k_1,k_2$ 构成的切平面,$z = k_1 x + k_2 y$(这里省略了一个常数 $C$,我们会假想 $C$ 使得恰好与二维上凸壳相切的,但其实整体平移没有影响) 

$$
g(k_1,k_2) := \max_{x,y}\left\{ f(x,y)-k_1(x-a) - k_2(y-b) \right\}
$$

应该能有 $g(k_1,k_2)$ 关于两维联合下凸,且最小值在 $(k_1,k_2) = \left( f'_x(a,b), f'_y(a,b) \right)$ 处取到。(或者离散情况下,可以是包含 $(a,b)$ 的某些平面的并)

联合下凸应该能保证,先预先对 fixed 的 $x$ 找完 $y$ 这维随便取后的最值,仍能在 $x$ 方向有凸性。

上面的内容的严谨性还待我进一步阅读专业书籍。总之网上存在一些并不那么合适的论述,不当弱化了一些条件。

此外实现的时候不要忘记凸性取反。

\lstinputlisting{Convex_Optimization/monge/wqs_binary_search2.cpp}