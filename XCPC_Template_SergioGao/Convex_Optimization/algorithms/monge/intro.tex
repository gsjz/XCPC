\section{四边形不等式相关}

要讨论的是这样的一类问题

$$
f(j) = \min_{i\le j}  w(i,j)
$$

关于某些 $w$ 满足四边形不等式的证明:

(i) 
把不等式写出来后,由于实质上只是左端点交换了一下,只与左端点有关的或者只与右端点有关的并不会改变,两边长得一样的,可以消去;常量更加可以消去了。
for example, 有时候 $w(i,j)$ 可能会被替换成 $w'(i,j) = dp[i-1] + w(i,j)$,然而 $dp[i-1]$ 这一项只与左端点有关,所以其是否满足四边形不等式并不会被改变。还有就像是 $\lambda(j-i+1)$ 这种可以被拆成左、右端点独立作用的也会被消去。

(ii)
通过一点小技巧,可以发现 $i<i+1\le j < j+1$ for $j=j_0,j_0+1$ 满足四边形不等式 $\implies$ $i<i+1\le j < j+2$ 满足四边形不等式,同理可以如此扩展左端点。
所以可以只证明左端点相差 $1$ 与右端点相差 $1$ 的情况。

(iii)
对于显式的、可求二阶导的下凸函数 $f$,$w(i,j) := f(j-i)$,证明则不需要这么麻烦。稍微推导可以发现无非是要说明对于 $a< b\le c< d$ at the same time $a+d = b+c$,which actually represent the length of the 4 mentioned intervals,we have $f(b) + f(c) \le f(a)+f(d)$。FYI, there's a special interpretation on "intersect $\le$ include", which is "equal $\le$ different"。这样的话其实就是要说明 $f(b) - f(a) \le f(d) - f(c)$,然而 $b-a = d-c$,从而就是要说明 $f(a+\Delta) - f(a) \le f(c+\Delta) - f(c)$。这几乎就是下凸性质的定义。

(iv)
\href{https://rebxe.github.io/post/jue-ce-dan-diao-xing-you-hua-dp-xue-xi-bi-ji/}{决策单调性优化 dp 学习笔记 | Exber's Blog}
上述性质似乎可以扩展(但没有提供证明),如果 $w(l,r)$ 满足区间包含单调性和四边形不等式,$f(x)$ 为下凸函数,则:
\begin{itemize}
    \item $f(w(l,r))$ 满足四边形不等式
    \item 如果 $f(x)$ 单调不降(也就是没有触底反弹那一段),则 $f(w(l,r))$ 还满足区间包含单调性
\end{itemize}

(v)
还有一些问题,在考虑四边形不等式进行代数运算时,不妨考虑对于三个区间 $[l_1,l_2],[l_2,r_1],[r_1,r_2]$ 分别设一些变量,这样代数证明会明显一点。比如说证明
$$
f(s_1+s_2) + f(s_2+s_3) \le f(s_1+s_2+s_3) + f(s_2)
$$

对于区间划分型 dp,\textbf{听说}恰经过 $k$ 条边蒙日矩阵最短路关于 $k$ 是下凸的。