% 整个文档的主文件。只需要在这个地方编译一次即可。

\documentclass[openany]{ctexbook}      
% 支持中文的书籍文档类
% [openany] 就是单面打印的意思。如果把这个去掉后,
% 则每次 include 都会自动空页来保证每章第一页都在奇数页。

\usepackage{pdfpages}



\usepackage{listings}
\usepackage{xcolor} %(可选,为了高亮配色)
\lstset{
  language=C++,               % 代码类型
  basicstyle=\ttfamily\footnotesize, % 代码字体小号等宽
  keywordstyle=\color{blue},  % 关键词蓝色
  commentstyle=\color{gray},  % 注释灰色
  numbers=none,               % 行号在左
  numberstyle=\tiny\color{gray},
  breaklines=true,            % 自动换行
  frame=single,               % 框出代码区域
  columns=fullflexible,
}
% 用 listings 来展示代码

\usepackage[
    hidelinks,         % 不标色也不加框,保持外观整洁
    bookmarksopen=true,% 打开时书签默认展开
    bookmarksnumbered=true, % 书签带章节号
    unicode=true       % 书签支持Unicode,防止中文乱码
]{hyperref}

\hypersetup{
    colorlinks=true,    % 使用彩色文字(非边框)
    linkcolor=blue,     % 内部链接颜色
    urlcolor=red,       % URL链接颜色
    citecolor=magenta    % 引用颜色
}

\usepackage[top=2cm, bottom=2cm, left=2cm, right=2cm]{geometry}
% ctexbook 默认边距太大,有点浪费资源。在这里调一下

\usepackage{amsmath}
% \implies 蜜汁不能用

\title{XCPC 算法竞赛模板}
\author{Sergio Gao}



\renewcommand\contentsname{\songti\zihao{2} 目\quad 录}
\ctexset{
    chapter = {
        format = \centering\zihao{2}\songti\bfseries, % 居中,宋体,large字号
        name = {第\space, \space 章},
        number = \arabic{chapter},
    }
}
\ctexset{
    section = {
        format = \centering\zihao{-2}\songti\bfseries, 
    }
}
\ctexset{
    subsection = {
        format = \centering\zihao{-3}\songti\bfseries, 
    }
}

\begin{document}

\maketitle
\tableofcontents



\section{初等数论相关}
\section{初等数论相关}
\include{Graph_Theory/tarjan/tarjan_scc}
\section{初等数论相关}
\include{Graph_Theory/flow/dinic_flow}
\include{Graph_Theory/flow/dijkstra_flow}

\section{初等数论相关}
\section{初等数论相关}
\subsection{四边形不等式优化 DP}

w 可替换任意符合四边形不等式:相交 <= 包含的 f(i,j)

\lstinputlisting{Convex_Optimization/algorithms/monge/monge_dp.cpp}
\subsection{预备节:WQS 二分}

\href{https://www.luogu.com.cn/problem/P2619}{P2619 [国家集训队]Tree I - 洛谷}

洛谷题解质量有点差,很多都看不出和 wqs 二分有什么关系。

这样思考,设 $g_x$ 是恰选中 $x$ 条白边的最小权。那么 $f(x) := g_x - \lambda x$ 就是给这 $x$ 条白边的权值减去 $\lambda$ 的惩罚代价后的权值,那么 $g_x-\lambda x$ 的最小值在什么地方取到?这等价于提前给所有白边都减去 $\lambda$,然后跑 mst。这有什么意义呢?

考虑 $(need-1, g_{need-1})$ 和 $(need, g_{need})$,当 $\lambda$ 恰好等于这两个点的斜率时,如果能有下凸性,并且同取最小值的时候可以取 $x$ 大的,这时候就会取到后面的那个作为最小值的 $x$ 点(如果取到 $>need$ 的其实同理可以往回取)。

当 $\lambda$ 大于这两个点的斜率时,$< need$ 的部分 比 大于等于 $\ge need$ 的部分不优。

当 $\lambda$ 大于这两个点的斜率时,$< need$ 的部分 比 大于等于 $\ge need$ 的部分更优。

所以二分这个惩罚代价,使得 $p = \max\{pos : f(pos) = \min_{x}\{f(x)\}\}$ 位于 $\ge need$ 部分,那么必然通过调整使得 $f(need) = f(p)$。

貌似有道更难一点的:\href{https://www.luogu.com.cn/problem/P5633}{P5633 最小度限制生成树 - 洛谷}。

\href{https://www.luogu.com.cn/problem/P1912}{P1912 [NOI2009] 诗人小G - 洛谷}

\href{https://www.luogu.com.cn/article/e362a4cs}{题解 P1912 【[NOI2009]诗人小G】 - 洛谷专栏}

\lstinputlisting{Convex_Optimization/monge/wqs_binary_search.cpp}
\subsection{WQS 二分结合决策单调性}

这个属于是固定套路,区间分划问题。

\lstinputlisting{Convex_Optimization/algorithms/monge/wqs_monge_dp.cpp}
\section{初等数论相关}
\subsection{李超线段树}

要求在平面直角坐标系下维护两个操作:

\begin{itemize}
    \item 在平面上加入一条线段。记第 $i$ 条被插入的线段的标号为 $i$。
    \item 给定一个数 $k$,询问与直线 $x = k$ 相交的线段中,交点纵坐标最大的线段的编号。
\end{itemize}

\lstinputlisting{Convex_Optimization/algorithms/Convex_Datastructure/lichao_tree.cpp}

\section{初等数论相关}
\section{初等数论相关}
\include{Computing_Geometry/random_methods/least_common_circle}

\section{初等数论相关}
\section{初等数论相关}
\include{Math/number_theory/primitive_root}
\subsection{杜教筛求 premu}


\lstinputlisting{Math/algorithms/number_theory/dujiao_sieve.cpp}

% \includepdf[pages=-]{Math/algorithms/number_theory/dujiao_sieve.pdf}


\section{初等数论相关}
\include{Math/polynomials/NTT}

% 我的文件管理规范:
% 对于大类,新建一个文件夹。
% 里面写一个 intro.tex,新起一个 chapter,顺便来介绍这个大类的总体内容
% 然后再建一个二级文件夹。
% 里面放各种算法的 .tex 和 .cpp
% 同理可以再在建三级文件夹。一般不建议四级以上


\end{document}


